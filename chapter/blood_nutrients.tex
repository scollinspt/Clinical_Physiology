% !TEX root = ../notes_template.tex
\chapter{Digestion-Absorption-Metabolism}\label{chp:blood_nutrients}
Updated on \today
\minitoc
This chapter covers the digestion and absorption from the viscera (gut) into the blood, and the metabolism of, the nutrients necessary in the blood to support the extra-cellular fluid. 



\vspace{5mm}

\textbf{Objectives include:}
\begin{enumerate}
    \item
    \item
    \item
    \item
    \item
\end{enumerate}

\section{Digestion - Absorption - Metabolism Overview}
Anatomy
Physiology
Regulation - enteric nervous system
Microbiome - 

Introduce these topics enough so that they can be included in other sections - - the importance of them in digestion and absorption


\section{Digestion}

Include TMJ for chewing?
Include dental health?
Major concepts - not details - 

\section{Absorption}

Major concepts - not details - 

Rates of absorption - performance under ideal situations and non ideal situations
Influence of blood flow on rates of absorption
Influence of GI conditions on rates of absorption
Influence of absorption on long term health

\subsection{Nutrients}

\subsubsection{Proteins}


\section{Metabolism}


\subsection{Pharmacokinetics}


\section{Gut Microbiome}

Probably needs to be introduced earlier - as part of an overview
This section should focus on the gut microbiome and the muscle axis


\section{\textit{Muscle Connections:}}

\subsection{Calcium Homeostasis}
Hypocalcemia, hyperreflexia, muscle cramps, numbness and tingling, twitches
Decreased ECF calcium levels, lowers the threshold potential making it easier to reach threshold for excitation

Hypercalcemia - at high concentrations calcium blocks sodium channels and inhibits depolarization of nerve and muscle fibers, increased calcium raises the threshold for depolarization.[5] This results in diminished deep tendon reflexes (hyporeflexia), and skeletal muscle weakness.

\subsection{Creatine Supplementation}

\printbibliography[heading=subbibintoc]