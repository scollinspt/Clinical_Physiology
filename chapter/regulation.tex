% !TEX root = ../notes_template.tex
\chapter{Muscle Regulation}\label{chp:regulation}

\minitoc
The regulation of muscle tension is managed by the central nervous system. This chapter introduces the basic mechanisms utilized by the CNS to fulfill this role. The CNS has one approach for muscle fiber tension regulation - it can manipulate the number of twitches per second (frequency summation). It also has one approach for whole muscle tension regulation - it can manipulate the number of muscle fibers twitching (motor unit summation). 

\vspace{5mm}

\textbf{Objectives include:}
\begin{enumerate}
    \item
    \item
    \item
    \item
    \item
\end{enumerate}

\section{Frequency Summation}

\section{Motor Unit Summation}

\subsection{Motor Units}

\subsubsection{Muscle Fiber Differentiation}

\subsection{Order of Recruitment}

\section{Regulatory Feedback}

\subsection{Muscle Spindles}

\subsection{Golgi Tendon Organs}

\section{\textit{Connections:} Electromyography}

\printbibliography[heading=subbibintoc]