% !TEX root = ../notes_template.tex
\chapter{Myostasis}\label{chp:myostasis}
Updated on \today
\minitoc
This chapter introduces readers to the physical stress theory \cite{mueller_tissue_2002}. It extends this theory with new insights into the hypertrophy, isotrophy and atrophy signal transduction pathways. It also covers the molecular epigentic basis for muscle memory and the emerging understanding of the gut microbiome - muscle axis. Adaptation (responsiveness to training) and plasticity (ability to adapt) based on genetic factors, and in response to interventions that do not cause injury are considered.

% Include basics of immunological function

\vspace{5mm}

\textbf{Objectives include:}
\begin{enumerate}
    \item
    \item
    \item
    \item
    \item
\end{enumerate}

\section{Muscle Development}

\section{Physical Stress Theory}

\section{Hypertrophy - Isotrophy - Atrophy Spectrum}

%How an activity influences muscle fatigue is based on the overload. The overload training concept is the stimulus for training related muscle adaptations. Training adaptations often occur in response to training fatigue in an attempt to build system capacity to avoid future fatigue. The relationship between fatigue and training adaptations is provided in Chapter \ref{chp:myostasis} on Myostasis.

\subsection{Signal Transduction Pathways}

\subsection{Transcription - Translation}

\subsection{Molecular Muscle Memory}

\subsection{Gut Microbiome - Muscle Axis}

\section{Adaptation \& Plasticity}

\subsection{Training Responses}

\subsection{Hypoxia}

\subsection{Blood Flow Restriction}

\printbibliography[heading=subbibintoc]