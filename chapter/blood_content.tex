% !TEX root = ../notes_template.tex
\chapter{Blood Volume support of Blood Flow}\label{chp:blood_content}
Updated on \today
\minitoc
This chapter covers the blood volume, all of the various components of blood that are necessary for it to fulfill its role in supporting cells via the correct extra-cellular fluid environment.

\vspace{5mm}

\textbf{Objectives include:}
\begin{enumerate}
    \item
    \item
    \item
    \item
    \item
\end{enumerate}

\section{Blood Content Overview}

Blood - 0.6 plasma, 0.4 red blood cells 

Hematocrit is fraction that is RBC
Anemia (death < 0.10)

\section{Renal Function}

\section{Electrolytes}

\section{Cells}

\subsection{Carrier Cells}

\subsection{Immunity}
% Role of lymph 
% Role of white blood cells

\section{Proteins}



\subsection{Sodium}

\subsection{Potassium}

\subsection{Calcium}

Hypocalcemia, hyperreflexia, muscle cramps, numbness and tingling, twitches
Decreased ECF calcium levels, lowers the threshold potential making it easier to reach threshold for excitation

Hypercalcemia - at high concentrations calcium blocks sodium channels and inhibits depolarization of nerve and muscle fibers, increased calcium raises the threshold for depolarization.[5] This results in diminished deep tendon reflexes (hyporeflexia), and skeletal muscle weakness.

\printbibliography[heading=subbibintoc]