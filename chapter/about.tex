% !TEX root = ../notes_template.tex
\chapter*{Preface}
\addcontentsline{toc}{chapter}{Preface}

% GitHub test

% 

Clinical Physiology: A muscle centered approach is a manuscript (book) in progress. The draft you are reading was released \today.

The project emerged during the authors' conversations about Unifying Systems Theory (UST) \cite{kahlen_perception_2017}. UST provided the system framework for the primary author (Collins) to consider how physical therapists should learn clinical physiology as a foundational knowledge for practice. 

We address questions about muscles as relevant for physical therapy practice. The line of questioning begins such as this: 

% Develop a format for dialogue
\vspace{5mm}

\noindent DPT Student: "I want practice physical therapy, what do I need to know about physiology?" 

\vspace{5mm}

\noindent Answer: "You need to know about muscle." 

\vspace{5mm}

\noindent To know about muscles, it's useful to think as a muscle fiber. 
\vspace{5mm}

\noindent "I am a muscle fiber. What do I do? How do I do it? What do I need? Is that all I need, or do I need more? What is necessary for me to keep doing what I do?"

\vspace{5mm}

\noindent These questions are addressed in this book. The reader is asked to "think as the muscle."

\subsection{Fitting Clinical Physiology: A muscle centered approach into a DPT curriculum}
A DPT program has the fundamental problem of attempting to educate individuals from varying backgrounds in the core set of knowledge and skills necessary to become licensed and to safely and ethically practice physical therapy. This cannot be done in one course. Learning and practicing the knowledge and skills must be spread across courses (during a term) and across time (from term to term) because of the volume.  The learning and practice then must build on prior learning and practice. The primary author of this book developed the curriculum that this book is suited. The course (Clinical Physiology) is one of two foundational courses offered in the first term.\footnotemark{}\footnotetext{Along with two practice courses and one systems course} The book fits best as a first term course, or a first course on human physiology.\footnotemark{}\footnotetext{The book does assume the reader has taken the typical pre requisite courses expected of most DPT programs such as anatomy \& physiology, physics, chemistry, biology.} It is not an exercise physiology text, though it has a lot of content that would be found in an exercise physiology text. It is not a medical physiology text, though it has a lot of content that would be found in a medical physiology text. The book includes foundational knowledge of physiology for a physical therapist. That knowledge falls in a previously unmet textbook space that combines exercise physiology and medical physiology. The primary author's experience includes teaching exercise physiology, medical physiology, exercise prescription, quantitative physiology, pathophysiology, pharmacology and physical therapy for people with cardiopulmonary and medically complicated conditions. The material in this book provides the foundational knowledge to practice physical therapy. This includes success in future courses, as well as by providing a unifying framework from which to consider or develop developments in practice. Physical therapists are human movement specialists. Volitional human movement requires muscle tension. Muscle tension requires physiological mechanisms and support. Therefore, physical therapists must be specialists in the physiological mechanisms and support of muscles.

 The approach taken to this topic (critical realist epistemology and Unifying Systems Theory) also asks readers to develop an understanding of how knowledge is obtained and applied for a knowledge based practice. Physical therapists are faced with uncertainty as they apply knowledge that has been generalized for a universal, abstract knowledge to particular situations \cite{collins_complex_2005}. The application of universal knowledge to a particular situation requires the ability to consider whether and how well the situation matches the universal knowledge) \cite{collins_synthesis_2018, collins_particulars_2018}. Depth and breadth of knowledge about physiological mechanisms helps us consider possible sources of variation and uncertainty in particular situations. For example, an early step in deciding whether the results of a clinical trial can be applied to a particular patient is considering whether the subjects in the trial are physiologically similar to the particular patient.


\subsection*{About the authors}

\begin{itemize}
\item Sean Collins is a physical therapist with a doctor of science in ergonomics and epidemiology from the UMass Lowell. He was professor of physical therapy and biomedical engineering at UMass Lowell for 18 years prior to relocating to Plymouth State University (PSU) as the founding director of the Doctor of Physical Therapy (DPT) program where is currently a professor. This project is possible because of the PSU DPT curriculum, and the generous sabbatical support from the PSU administration and DPT faculty colleagues.
\item Bog Sniezek is a systems scientist
\item Tom Sniezek is a physical therapist, graduate of the DPT program at UMass Lowell and currently practices physical therapy in Ohio
\item Nathaniel Mailloux is a physical therapist, graduate of the DPT program at Plymouth State University and currently practices physical therapy in New York
\end{itemize}

\printbibliography[heading=subbibintoc]



