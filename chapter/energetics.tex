% !TEX root = ../notes_template.tex
\chapter{Muscle Energetics}\label{chp:energetics}
Updated on \today
\minitoc

All prior chapters include important molecular processes for muscle function that require the use of adenosine tri-phosphate (ATP) to provide required energy. Even the resting state of the muscle fiber is in continuous need for ATP to maintain the membrane potential. This chapter covers the biochemical pathways utilized by muscle fibers to convert fuel (macro-nutrients) into ATP.

\vspace{5mm}

\textbf{Objectives include:}
\begin{enumerate}
    \item Explain the structure and function of energetic components of muscle fibers.
    \item Compare and contrast the energetic system pathways that transform macro nutrient substrates to ATP.
    \item Compare and contrast muscle fiber differentiation (types) based on energetic structures and resultant functional capacities.
    \item  Apply the concepts of mass balance, flow gradients, and energy to the analysis of patient/client problems related to the muscular system.
     \item Evaluate the different energetic pathways used in different activities, and analyze the response to different activities to energetic pathways.
     \item Explain the energetic basis of fatigue.
     \item Explain the energetic basis of ischemia.
\end{enumerate}

\section{Energetic Transformation Overview}

% Use of ATP (muscles use ATP to convert chemical energy to mechanical energy (Myosin ATPase)

ATP promotes three types of cell function: (1) membrane transport, as occurs with the sodium-potassium pump, which transports sodium out of the cell and potassium into the cell; (2) synthesis of chemical compounds throughout the cell; and (3) mechanical work, as occurs with the contraction of muscle fibers or with ciliary and ameboid motion.


% Regular regeneration - we don't store a lot - include why

% No switch - all of the biochemical pathways are occuring at the same time, the flux through each is dependent on the need for ATP and that need is based on the need for energy. High metabolism refers to the situation of needing more ATP. The entire system of using $O_2$ consumption as a biometric of metabolism is predicated on the ultimate use of $O_2$ to energize all ATP. Even when PCr is used to energize a muscle activation the creatine (Cr) is ultimately regenerated (PCr) by an ATP 

\section{ATP Regeneration Pathways}

If $CO_2$ is created it is cellular respiration as part of the citric acid cycle (TCA). $O_2$ is consumed as part of the electron transport chain (ETC). Note that $CO_2$ is produced and $O_2$ is consumed at different steps in the pathways that produce $CO_2$ a byproduct and consume $O_2$.

\subsection{ATP / Phosphocreatine (PCr) (immediate) Pathway}

% Include the concept of diffusion of ATP through the muscle cell and the hypothesized use of PCr to facilitate the transport of ATP through the sarcoplasm

Creatine is synthesized in the liver, or ingested, digested and then absorbed from meat or ingested and absorbed from supplementation.\footnotemark\footnotetext{Creatine is a molecule that can be absorbed directly through the gut into the blood. Creatine supplementation will be a clinical connection in Chapter \ref{chp:blood_nutrients} on Visceral Support.}

$PCr \rightleftharpoons ATP$

Back and forth, high rate reaction catalyzed by the enzyme creatine kinase (CK)

At rest about 80\% of creatine is in the energized (phosphoralated) form of PCr, and there is approximately five times as much PCr than ATP (5:1 ratio).

The $PCr \rightleftharpoons ATP$ is high rate it can generate upwards of 4.4 moles/minute of ATP. However, it is only capable of regenerating approximately 0.7 moles of ATP overall. Therefore, when working at maximum capability (rate of 4.4 moles/minute) it would only be sustained for 9.5 seconds (See the Summary of Max Rates of ATP Regeneration by Pathway in Table \ref{table:ATP_Rates}. This time estimate is dependent on the assumption of the max rate (4.4 moles/minute). If a lower rate of ATP was being utilized, say .5 moles/minute, then PCr would be able to sustain ATP regeneration much longer. And if that rate of ATP utilization was lower than other pathways then PCr would continue, itself, to be regenerated and the need for ATP from PCr would not approach maximum because other pathways would be contributing to the ATP need.

The PCr pathway is also hypothesized to be utilized as a ATP transport mechanism throughout the sarcoplasm (PCr shuttle). There are substantial diffusion barriers in a muscle fiber due to the number of highly structured protein complexes. The mitochrondria are near the sarcolemma (along with the nuclei largely because the rest of the fiber is packed tightly with myofilaments of sarcomeres) and thus the diffusion of ATP from the mitochrondria encounters many barriers.  PCr is diffused through the entire sarcoplasm and due to its rapid regeneration rate can easily shuttle (or pass) the energy of ATP through the sarcoplasm from just outside a mitochondria to near the myosin ATPase. During muscle activation when the rate of ATP use by the myosin ATPase increases the nearly PCr regenerate and then that regeneration can propagate from all around the sarcoplasm to move ATP toward the myosin ATPase (See Figure \ref{fig:PCr}). This movement - either the diffusion of ATP directly or the use of a PCr shuttle would be, of course, based on the flow gradient. The more ATP being utilized by the myosin ATPase, the $Na^+/K^+$-ATPase or the $Ca^{2+}$ ATPase the lower the concentration of ATP near those proteins and the greater the diffusion, and shuttling, of ATP towards those locations in the sarcoplasm.

\begin{figure}[h!]
    \centering
    \includegraphics{}
    \caption{Movement of ATP through the sarcoplasm using a PCr shuttle}
    \label{fig:PCr}
\end{figure}

\paragraph{Adenylate Cyclase}

Another "immediate" high rate reaction to regenerate ATP is the adenylate cyclase (an enzyme) that catalyzes the reaction: $ADP \rightleftharpoons AMP + ATP$. During this reaction a $P_i$ is hydrolyzed from ADP and the energy is utilized to regenerate ATP. While this reaction can happen at a high rate, it is limited and probably only utilized in extreme "ATP need" situations. In such situations this reaction allows continued cell functions for a short period of time. 

\subsection{Glycolosis $\rightarrow$ Lactate Pathway}

\subsection{Muscle Glycogen $\rightarrow$ $CO_2$ Pathway}

$CO_2$ produced as part of TCA (also referred to as Kreb's cycle)
$O_2$ consumed as part of ETC (also referred to as oxidative phosphorylation)
TCA and ETC combined can be referred to as oxidative metabolism. 

Even though the $CO_2$ production and the $O_2$ consumption does not occur at the same steps in the pathways, whenever you see that $CO_2$ is produced, you can assume that $O_2$ is being consumed. 

\subsection{Liver Glycogen $\rightarrow$ $CO_2$ Pathway}

\subsection{Fatty Acids $\rightarrow$ $CO_2$ Pathway}



Mitochondria Extract Energy From Nutrients (p. 24) The principal substances from which the cells extract energy are oxygen and one or more of the foodstuffs—carbohydrates, fats, and proteins—that react with oxygen. In humans, almost all carbohydrates are converted to glucose by the digestive tract and liver before they reach the cell; similarly, proteins are converted to amino acids, and fats are converted to fatty acids. Inside the cell, these substances react chemically with oxygen under the influence of enzymes that control the rates of reaction and channel the released energy in the proper direction.

Mitochondria Extract Energy From Nutrients (p. 24) The principal substances from which the cells extract energy are oxygen and one or more of the foodstuffs—carbohydrates, fats, and proteins—that react with oxygen. In humans, almost all carbohydrates are converted to glucose by the digestive tract and liver before they reach the cell; similarly, proteins are converted to amino acids, and fats are converted to fatty acids. Inside the cell, these substances react chemically with oxygen under the influence of enzymes that control the rates of reaction and channel the released energy in the proper direction.
Oxidative Reactions Occur Inside the Mitochondria, and Energy Released Is Used to Form ATP ATP is a nucleotide composed of the nitrogenous base adenine, the pentose sugar ribose, and three phosphate radicals. The last two phosphate radicals are connected with the remainder of the molecule by high-energy phosphate bonds, each of which contains about 12,000 calories of energy per mole of ATP under the usual conditions of the body. The high-energy phosphate bonds are labile so they can be split instantly whenever energy is required to promote other cellular reactions. When ATP releases its energy, a phosphoric acid radical is split away, and adenosine diphosphate (ADP) is formed. Energy derived from cell nutrients causes ADP and phosphoric acid to recombine to form new ATP, with the entire process continuing over and over again. Most of the ATP Produced in the Cell Is Formed in Mitochondria After entry into the cells, glucose is subjected to enzymes in the cytoplasm that convert it to pyruvic acid, a process called glycolysis. Less than 5\% of ATP formed in the cell occurs via glycolysis. Pyruvic acid derived from carbohydrates, fatty acids derived from lipids, and amino acids derived from proteins are all eventually converted to the compound acetyl coenzyme A (acetyl-CoA) in the mitochondria matrix. This substance is then acted on by another series of enzymes in a sequence of chemical reactions called the citric acid cycle, or Krebs cycle. In the citric acid cycle, acetyl-CoA is split into hydrogen ions and carbon dioxide. Hydrogen ions are highly reactive and eventually combine with oxygen that has diffused into the mitochondria. This reaction releases a tremendous amount of energy, which is used to convert large amounts of ADP to ATP. This requires large numbers of protein enzymes that are integral parts of the mitochondria. The initial event in ATP formation is removal of an electron from the hydrogen atom, thereby converting it to a hydrogen ion. The terminal event is movement of the hydrogen ion through large globular proteins called ATP synthetase, which protrude through the membranes of the mitochondrial membranous shelves, which themselves protrude into the mitochondrial matrix. ATP synthetase is an enzyme that uses the energy from movement of the hydrogen ions to convert ADP to ATP, and hydrogen ions combine with oxygen to form water. The newly formed ATP is transported out of the mitochondria to all parts of the cell cytoplasm and nucleoplasm, where it is used to energize the functions of the cell. This overall process is called the chemiosmotic mechanism of ATP formation.



\subsubsection{Pathways for Protein Energetics}



\subsection{Rates and Capacities}

\begin{table}[h!]
\centering
\begin{tabular}{||c c c c||} 
 \hline
Source & Max Rate of ATP (mol/min) & Amount of ATP (mol) & Time at Max (s or min)\\ [0.5ex] 
 \hline\hline
 ATP/PCr & 4.4  & 0.7 & 9.5 s \\
 Glycolosis $\rightarrow$ Lactate &  2.4 & 1.6 & 40 s \\ 
 Muscle Glycogen $\rightarrow$ $CO_2$ & 1.0 & 84 & 84 min\\
 Liver Glycogen $\rightarrow$ $CO_2$  & 0.4 & 19 & 48 min \\ 
 Fatty Acids $\rightarrow$ $CO_2$ & 0.4 & 4000 & 10,000 min \\[1ex] 
 \hline
\end{tabular}
\caption{Summary of Max Rates of ATP Regeneration by Pathway (\footnotesize{Data from \cite{feher_quantitative_2017}})}
\label{table:ATP_Rates}
\end{table}

When considering the time estimates based on the quantity of ATP can be be regenerated by various energetic pathways it is important to consider that two factors influence the sustainability of the pathway. First whether the pathway, at its max rate, is rate limiting. Second, the availability of the substrate (macro-nutrients) in storage. Another consideration for the time estimates in Table \ref{table:ATP_Rates} is that the times are calculated based on the assumption of the max rate. For example, the 10 second estimate for ATP/PCr (which is a commonly cited estimate for this system) is entirely predicated on the max rate of ATP regeneration from ATP/PCr, which is based on the demand for ATP by the muscle. If the demand for ATP by the muscle is 2.2 mol/min then the time that ATP/PCr could contribute would be approximately 19 seconds. If the demand for ATP by the muscle is 1.1 mol/min then the time that ATP/PCr could contribute would be approximately 38 seconds. By this time the Muscle Glycogen $\rightarrow$ $CO_2$ pathway could be contributing and meeting most of the ATP demand of the muscle. 

An important point is that there is no such thing as ATP dept. Energy is needed for cellular functions. Either the pathways provide a way to regenerate the ATP needed for the demand, or the cellular functions do not occur.


\section{Motor Unit \& Muscle Fiber Types}

\subsection{Slow Oxidative (S/SO/Type 1}

\subsection{Fast Oxidative Glycolytic (FR/FOG/Type 2x}

\subsection{Fast Glycolytic (FF/FG/Type 2a}




\section{\textit{Clinical Physiology Connections}}

\subsection{Sarcopenia}

Sarcopenia is an age related loss of muscle fibers with a bias toward FG fibers (FF motor units). A consequence of sarcopenia is a loss of the ability to attain the upper ranges of muscle tension. Expected consequences are a reduction in peak tension, high forces and high velocities of movement. An unexpected, paradoxical, consequence is a reduction in the ability to attain lower ranges of muscle tension. This paradoxical consequence is based on the need to recruit the remaining motor units with increased frequency to meet the tension requirements of movement. Recruiting an S motor unit and its SO fibers more frequently (frequency summation) requires a higher rate of ATP than can be sustained. There is no question that S motor units and SO muscle fibers can achieve a higher rate of ATP than FF motor units and FG muscle fibers. But at a high level of tension (relative to the motor units capacity) the higher rate of crossbridge activation requires a higher rate of ATP regeneration. If that higher rate of ATP regeneration is greater than the rate capacity of the Glycolosis $\rightarrow$ $CO_2$ pathway, then the fiber will need to utilize the Glycolosis $\rightarrow$ Lactate pathway, which is self limiting (cannot be sustained). 

Let's walk through it with a simple model example. Assume walking requires 50\% output (tension) of 1/4 of the motor units. At this output the fibers require 0.7 moles of ATP / minute. In this scenario the Glycolosis $\rightarrow$ $CO_2$ pathway is sufficient and the activity is sustainable. Now assume there has been a loss of motor units. Now there are 1/2 of the prior motor units, and they are predominantly lower tension generating (S/SO) because the loss (due to sarcopenia) results in a loss of FG/FOG muscle fibers. Now walking (at the same pace as before) requires 75\% output from 3/4 of the fibers (since a twitch generates less tension there needs to be more frequency and motor unit summation). At this output the fibers require 1.3 moles of ATP / minute. In this scenario the Glycolosis $\rightarrow$ $CO_2$ pathway is not sufficient and the Glycolosis $\rightarrow$ Lactate pathway must be utilized to supplement ATP regeneration, which is not sustainable. Continuing to walk at this pace will result in fatigue. That is the paradoxical consequence of the loss of FF motor units in sarcopenia. Everyone expects a loss in the ability to attain tension (as measured by peak force). But there is also a loss in the ability to sustain tension (as measured by the ability to continue developing tension for an activity).


\subsection{Fatigue}

\subsection{Ischemia, Hypoxemia \& Hypoxia}

The oxidative energetic pathways provide most of the ATP for cellular functions and are critically involved in the restoration of the energetic resting state following short term rate increases in ATP production with CP or glycolosis. These pathways require oxygen and macro-nutrient substrate (seconds-minutes). Over longer time periods (hours-days) they require cellular synthesis of enzymes which require ATP and amino-acids (to build proteins). And over longer time periods (days-weeks) they require maintenance of the health of mitochondria and replacing damaged or dead mitochondria. Interruptions in the availability of O2 interrupts the process of ATP production with these pathways and can lead to the accumulation of metabolic waste products that alter the pH of the cell and the extra-cellular fluid. These interruptions form the basis of a large number of relatively common and life-threatening chronic medical conditions such as heart disease, stroke, peripheral vascular disease, COVID-19, pulmonary disease, and hematologic (blood) conditions; as well as sudden onset (acute) conditions such as a heart attack and acute altitude sickness. 
Some of the conditions have a rapid onset and immediately threaten life; others exert their effect gradually. The variation is related to rate of onset of O2 deprivation, the magnitude of O2 deprivation (how much deficit, how many cells), and whether the impact is just in the availability of O2 or whether there is also an impairment in waste product removal. But they can all be analyzed based on an understanding of cellular energetics and the role that ATP plays in cellular fidelity, efficacy and integrity.

\paragraph{Hypoxia}
Hypoxia refers to the situation in which there is not enough O2 getting to cells for them to sustain the oxidative production of ATP. Hypoxia can be caused by a variety of situations. It is a local condition because it depends on local O2 levels and local O2 needs. Local O2 needs are dependent on local metabolism. Cells of the body that have relatively high and constant O2 needs, and are therefore more susceptible to hypoxia, are the heart and brain. While metabolism is always related to cellular activity, these two organs tend to have a higher resting metabolism than other body cells. For example, muscle metabolism can far exceed that of both heart and brain, that is only during periods of high muscle activity which require high levels of ATP production. 
The two primary causes of hypoxia are hypoxemia and ischemia.

\paragraph{Hypoxemia}
Hypoxemia is a specific situation in which the blood isn't carrying adequate oxygen to the body’s tissues. Common causes of hypoxemia include pulmonary and blood conditions (for example, obstructive pulmonary disease and anemia) or environmental conditions (altitude, carbon monoxide). 

Hypoxemia can cause hypoxia. The severity of hypoxia in a cell caused by hypoxemia is dependent on the severity of hypoxemia as well as the metabolic activity of the cell (O2 demand), which fluctuates based on several factors. In someone with mild hypoxemia there may be no hypoxia in cardiac or skeletal muscles. However, with exertion that increases cardiac and skeletal muscle activity and thus need for O2 (O2 demand) there may be hypoxia. In these situations cardiac and skeletal muscle function will be impaired by the lack of O2. The cellular adjustment will be to provide ATP using a higher rate of CP and glycolosis which is not sustainable. The by-products of these activities will decrease the cellular and extra-cellular pH which will further impair the production of ATP. The acidosis and reduced ATP relative to need can impact the ability of the cells to repolarize, reduce the frequency of excitations, reduce the pumping of Ca+ back into the sarcoplasmic reticulum, reduce the rate of myosin head release. The consequences of these changes include lower tension production, spasm and potentially damage to the cell membrane. But typically, if the problem originates with hypoxemia that is adequate for resting levels of O2 demand then simply ceasing the activity will restore balance and not result in damage to the cell membrane.
 
\paragraph{Ischemia}
Ischemia is a specific situation in which  blood supply to cells is reduced. The extent of cellular involvement depends on the extent of the reduction. For example, if an entire artery is impacted than an entire limb, or muscle can be involved. If the reduction occurs in capillaries then the reduction in blood flow is to far fewer cells. Common causes of ischemia are atherosclerosis, arteriosclerosis,\footnotemark\footnotetext{Arteriosclerosis refers to thick and stiff arteries that can restrict blood flow. Atherosclerosis is a type of arteriosclerosis that includes buildup of fats, cholesterol and other substances in and on artery walls (plaque). The subsequent reduction in vessel diameter can limit blood flow, and if a plaque becomes an embolus it can lodge and completely block blood flow (blood clot).} blood clots (arterial thrombosis or embolus, or in the case of pulmonary blood flow venous thrombosis or embolus), blood vessel spasm, and micro-circulatory inflammation (a clinical manifestation of hypoxia itself and seen in COVID-19). 

Ischemia can cause hypoxia. The severity of hypoxia in a cell caused by ischemia is dependent on the severity of ischemia as well as the metabolic activity of the cell (O2 demand). A sudden and complete blockage of blood flow to cells, with no alternative pathways to provide blood flow to the cell, is a serious situation that results in cellular death due to the inability to produce ATP for cell membrane functions (sudden complete hypoxia) and to remove waste products. The combination of these two situations results first in reversible damage to the cell membrane and then to irreverisble damage to the cell membrane. Without the cell membrane the cell has lost its integrity. Less extreme reductions in blood flow can create a wide variety of hypoxic and waste removal situations that allow sustained but reduced function for a cell (resulting in long term problems in cell maintenance), and reduced function of the cells. For example, a limit on how much activity the cardiac or skeletal muscle can perform prior to having hypoxia. It is common to simply refer to such situations as ischemia (which means local blood flow is reduced and O2 demands are high enough to cause hypoxia. 

Given the variable nature of blood flow supply and cellular O2 demand there are two situations that can arise for cardiac muscle cells in particular. Stable ischemia refers to the situation that blood flow is sufficient for resting O2 demand, but not sufficient during elevated O2 demand (increased cardiac muscle activity). The situation is considered stable because simply reducing cardiac activity will reduce cardiac O2 demand and restore balance to allow recovery. Unstable ischemia refers to the situation that blood flow is not sufficient for resting O2 demand. Unstable ischemia is unstable because balance cannot be restored by reducing the cardiac muscle activity back to rest since it is the resting demand that cannot be met. In such situations blood flow must be restored, or cardiac muscle O2 demand must be reduced below resting levels. Restoring blood flow is highly situational and can involve breaking up a blood clot (thrombus) with medications (thrombolytics); or restoring the diameter of blood vessels with an angioplasty or by re routing the blood (by-pass graft. Reducing cardiac muscle O2 demand can be accomplished by lowering blood pressure (blood pressure is the resistance that the cardiac muscle must work). Nitroglycerine is a very powerful and fast dilator of blood vessels that quickly lowers blood pressure and allows the cardiac O2 demand to be lowered below resting values. The hope is this restores balance between O2 supply and O2 demand and the cells to recover before damage.

\paragraph{Summary}
Hypoxia is the basis of the homeostatic imbalances caused by many conditions and diseases. It is based on the cellular requirements for ATP, which are based on the mitochrondrial requirements for O2. Cells can produce ATP without O2, however they cannot sustain the production of ATP without O2. Understanding these mechanisms and that of hypoxia offers a wellspring of conceptual insights for many diseases and pathophysiological conditions that go well beyond the above discussion. Hypoxia caused by hypoxemia or ischemia continues to be topic for several of the upcoming chapters on Muscle Support.



\section{Summary \& Next Steps}




\printbibliography[heading=subbibintoc]